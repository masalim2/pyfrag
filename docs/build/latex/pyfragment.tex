%% Generated by Sphinx.
\def\sphinxdocclass{report}
\documentclass[letterpaper,10pt,english]{sphinxmanual}
\ifdefined\pdfpxdimen
   \let\sphinxpxdimen\pdfpxdimen\else\newdimen\sphinxpxdimen
\fi \sphinxpxdimen=.75bp\relax

\usepackage[utf8]{inputenc}
\ifdefined\DeclareUnicodeCharacter
  \DeclareUnicodeCharacter{00A0}{\nobreakspace}
\fi
\usepackage{cmap}
\usepackage[T1]{fontenc}
\usepackage{amsmath,amssymb,amstext}
\usepackage{babel}
\usepackage{times}
\usepackage[Bjarne]{fncychap}
\usepackage{longtable}
\usepackage{sphinx}

\usepackage{geometry}
\usepackage{multirow}
\usepackage{eqparbox}

% Include hyperref last.
\usepackage{hyperref}
% Fix anchor placement for figures with captions.
\usepackage{hypcap}% it must be loaded after hyperref.
% Set up styles of URL: it should be placed after hyperref.
\urlstyle{same}
\addto\captionsenglish{\renewcommand{\contentsname}{Table of Contents}}

\addto\captionsenglish{\renewcommand{\figurename}{Fig.}}
\addto\captionsenglish{\renewcommand{\tablename}{Table}}
\addto\captionsenglish{\renewcommand{\literalblockname}{Listing}}

\addto\extrasenglish{\def\pageautorefname{page}}

\setcounter{tocdepth}{1}



\title{pyfragment Documentation}
\date{Feb 28, 2017}
\release{0.1}
\author{Misha Salim}
\newcommand{\sphinxlogo}{}
\renewcommand{\releasename}{Release}
\makeindex

\begin{document}

\maketitle
\sphinxtableofcontents
\phantomsection\label{\detokenize{index::doc}}



\chapter{Background}
\label{\detokenize{background::doc}}\label{\detokenize{background:pyfragment-documentation}}\label{\detokenize{background:background}}
PyFragment is a collection of Python modules that facilitate the setup and
parallel execution of \sphinxstyleemphasis{embedded-fragment} calculations on molecular clusters,
liquids, and solids.


\section{Theory}
\label{\detokenize{background:theory}}
The \sphinxstyleemphasis{embedded-fragment} methods are rooted in the many body expansion (MBE),
which expresses the total energy of a molecular system as
\begin{equation*}
\begin{split}E = \sum_{i} E_{i}\end{split}
\end{equation*}\index{energy\_driver() (in module drivers)}

\begin{fulllineitems}
\phantomsection\label{\detokenize{background:drivers.energy_driver}}\pysiglinewithargsret{\sphinxcode{drivers.}\sphinxbfcode{energy\_driver}}{}{}
SP energy

\end{fulllineitems}



\section{Implementation}
\label{\detokenize{background:implementation}}
A series of modules are available. Please check:


\subsection{Codes}
\label{\detokenize{background:module-ChargeState}}\label{\detokenize{background:codes}}\index{ChargeState (module)}\index{ChargeState (class in ChargeState)}

\begin{fulllineitems}
\phantomsection\label{\detokenize{background:ChargeState.ChargeState}}\pysiglinewithargsret{\sphinxstrong{class }\sphinxcode{ChargeState.}\sphinxbfcode{ChargeState}}{\emph{fragments}, \emph{fragment\_charges}}{}
Base class for VB CT state
\index{coupling\_dimer\_gs() (ChargeState.ChargeState method)}

\begin{fulllineitems}
\phantomsection\label{\detokenize{background:ChargeState.ChargeState.coupling_dimer_gs}}\pysiglinewithargsret{\sphinxbfcode{coupling\_dimer\_gs}}{\emph{state2}, \emph{embed\_flag=None}}{}
-sqrt( {[}E\_(AB)+ - E\_(A+B){]}*{[}E\_(AB)+ - E\_(AB+){]})
E\_(AB)+ == the relaxed, correlated charged dimer
E\_(A+B) == non-stationary HF energy of the charge-local dimer
\begin{quote}

plus monomer correlation energies of E\_A+ and E\_B
\end{quote}

For a dimer system, this method reproduces the exact E by construction

\end{fulllineitems}

\index{coupling\_dimer\_gs\_no\_embed() (ChargeState.ChargeState method)}

\begin{fulllineitems}
\phantomsection\label{\detokenize{background:ChargeState.ChargeState.coupling_dimer_gs_no_embed}}\pysiglinewithargsret{\sphinxbfcode{coupling\_dimer\_gs\_no\_embed}}{\emph{state2}}{}
Wraps the above dimer\_gs coupling, but the two monomers do not
polarize each other in monomerSCF. Hence, it does not exactly
reproduce the dimer GS energy by construction

\end{fulllineitems}

\index{coupling\_dimer\_gs\_overlapHOMO() (ChargeState.ChargeState method)}

\begin{fulllineitems}
\phantomsection\label{\detokenize{background:ChargeState.ChargeState.coupling_dimer_gs_overlapHOMO}}\pysiglinewithargsret{\sphinxbfcode{coupling\_dimer\_gs\_overlapHOMO}}{\emph{state2}}{}
S*E\_(AB)+ - sqrt( {[}E\_(AB)+ - E\_(A+B){]}*{[}E\_(AB)+ - E\_(AB+){]})
Second term is same as coupling\_dimer\_gs
First term is overlap of monomer HOMO's times relaxed dimer energy
This method also reproduces energy of a dimer by construction; it
just makes the overlap matrix non-identity

\end{fulllineitems}

\index{diag\_chargelocal\_dimers() (ChargeState.ChargeState method)}

\begin{fulllineitems}
\phantomsection\label{\detokenize{background:ChargeState.ChargeState.diag_chargelocal_dimers}}\pysiglinewithargsret{\sphinxbfcode{diag\_chargelocal\_dimers}}{\emph{subcomm=None}}{}
return E1 + E2, the BIM energy of this charge-transfer configuration
E1: sum of monomer correlated energies
E2: sum of dimer interaction energies (E\_AB - E\_A - E\_B)
\begin{quote}

interaction INCLUDES correlation for relaxed dimers (when A\&B have same charge)
but it's just the non-stationary HF energy for charge-local dimers
\end{quote}

\end{fulllineitems}


\end{fulllineitems}



\chapter{Indices and tables}
\label{\detokenize{index:indices-and-tables}}\begin{itemize}
\item {} 
\DUrole{xref,std,std-ref}{genindex}

\item {} 
\DUrole{xref,std,std-ref}{modindex}

\item {} 
\DUrole{xref,std,std-ref}{search}

\end{itemize}


\renewcommand{\indexname}{Python Module Index}
\begin{sphinxtheindex}
\def\bigletter#1{{\Large\sffamily#1}\nopagebreak\vspace{1mm}}
\bigletter{c}
\item {\sphinxstyleindexentry{ChargeState}}\sphinxstyleindexpageref{background:\detokenize{module-ChargeState}}
\end{sphinxtheindex}

\renewcommand{\indexname}{Index}
\printindex
\end{document}