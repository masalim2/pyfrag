%% Generated by Sphinx.
\def\sphinxdocclass{report}
\documentclass[letterpaper,10pt,english]{sphinxmanual}
\ifdefined\pdfpxdimen
   \let\sphinxpxdimen\pdfpxdimen\else\newdimen\sphinxpxdimen
\fi \sphinxpxdimen=.75bp\relax

\usepackage[utf8]{inputenc}
\ifdefined\DeclareUnicodeCharacter
  \DeclareUnicodeCharacter{00A0}{\nobreakspace}
\fi
\usepackage{cmap}
\usepackage[T1]{fontenc}
\usepackage{amsmath,amssymb,amstext}
\usepackage{babel}
\usepackage{times}
\usepackage[Bjarne]{fncychap}
\usepackage{longtable}
\usepackage{sphinx}

\usepackage{geometry}
\usepackage{multirow}
\usepackage{eqparbox}

% Include hyperref last.
\usepackage{hyperref}
% Fix anchor placement for figures with captions.
\usepackage{hypcap}% it must be loaded after hyperref.
% Set up styles of URL: it should be placed after hyperref.
\urlstyle{same}
\addto\captionsenglish{\renewcommand{\contentsname}{Table of Contents}}

\addto\captionsenglish{\renewcommand{\figurename}{Fig.}}
\addto\captionsenglish{\renewcommand{\tablename}{Table}}
\addto\captionsenglish{\renewcommand{\literalblockname}{Listing}}

\addto\extrasenglish{\def\pageautorefname{page}}

\setcounter{tocdepth}{1}



\title{pyfragment Documentation}
\date{Feb 28, 2017}
\release{0.1}
\author{Misha Salim}
\newcommand{\sphinxlogo}{}
\renewcommand{\releasename}{Release}
\makeindex

\begin{document}

\maketitle
\sphinxtableofcontents
\phantomsection\label{\detokenize{index::doc}}



\chapter{Background}
\label{\detokenize{background::doc}}\label{\detokenize{background:pyfragment-documentation}}\label{\detokenize{background:background}}
PyFragment is a collection of Python modules that facilitate the setup and
parallel execution of \sphinxstyleemphasis{embedded-fragment} calculations on molecular clusters,
liquids, and solids.


\section{Theory}
\label{\detokenize{background:theory}}
The \sphinxstyleemphasis{embedded-fragment} methods are rooted in the many body expansion (MBE),
which expresses the total energy of a molecular system as (under
construction...)
\begin{equation*}
\begin{split}E = \sum_{i} E_{i}\end{split}
\end{equation*}

\section{Implementation}
\label{\detokenize{background:implementation}}
Under construction...


\subsection{Codes}
\label{\detokenize{background:codes}}
Under construction


\chapter{The \sphinxstylestrong{globals} Module}
\label{\detokenize{globals::doc}}\label{\detokenize{globals:the-globals-module}}
\sphinxstylestrong{globals} contains essential shared data and functionality for \sphinxstyleemphasis{all} types of fragment calculations.


\section{globals.geom}
\label{\detokenize{globals:globals-geom}}
Defines the fundamental Atom class to conveniently load and print geometry
information.  Contains functions for loading geometry and performing
\sphinxstyleemphasis{fragmentation}, that is, assigning which atoms belong to which fragments.
\phantomsection\label{\detokenize{globals:module-globals.geom}}\index{globals.geom (module)}\index{Atom (class in globals.geom)}

\begin{fulllineitems}
\phantomsection\label{\detokenize{globals:globals.geom.Atom}}\pysiglinewithargsret{\sphinxstrong{class }\sphinxcode{globals.geom.}\sphinxbfcode{Atom}}{\emph{atomstr}, \emph{units='angstrom'}}{}
Convenience class for loading and storing geometry data

\end{fulllineitems}

\index{load\_geometry() (in module globals.geom)}

\begin{fulllineitems}
\phantomsection\label{\detokenize{globals:globals.geom.load_geometry}}\pysiglinewithargsret{\sphinxcode{globals.geom.}\sphinxbfcode{load\_geometry}}{\emph{data}, \emph{units='angstrom'}}{}
Builds geometry from input text, lists, or filename.

Tries to be flexible with the form of input `data' argument. 
Uses regex to extract atomic coordinates from text.
\begin{description}
\item[{Args:}] \leavevmode\begin{description}
\item[{data: string, list of strings, list of lists, or filename}] \leavevmode
containing the xyz coordinate data

\end{description}

units (default Angstrom): ``bohr'' or ``angstrom''

\item[{Returns:}] \leavevmode
geometry: a list of Atom objects

\end{description}

\end{fulllineitems}

\index{makefrag\_auto() (in module globals.geom)}

\begin{fulllineitems}
\phantomsection\label{\detokenize{globals:globals.geom.makefrag_auto}}\pysiglinewithargsret{\sphinxcode{globals.geom.}\sphinxbfcode{makefrag\_auto}}{\emph{geometry}}{}
Auto-generate list of fragments based on bond-length frag\_cutoffs.

Use this if you don't wish to manually assign atoms to fragments.
\begin{description}
\item[{Args:}] \leavevmode
geometry: list of Atom objects

\item[{Returns:}] \leavevmode
fragments: a list of fragment atomic indices

\end{description}

\end{fulllineitems}

\index{makefrag\_full\_system() (in module globals.geom)}

\begin{fulllineitems}
\phantomsection\label{\detokenize{globals:globals.geom.makefrag_full_system}}\pysiglinewithargsret{\sphinxcode{globals.geom.}\sphinxbfcode{makefrag\_full\_system}}{\emph{geometry}}{}
No fragmentation: all atoms in system belong to one fragment.

Use this to perform one big reference QM calculation
\begin{description}
\item[{Args:}] \leavevmode
geometry: list of Atom objects

\item[{Returns:}] \leavevmode
fragments: a list of fragment atomic indices

\end{description}

\end{fulllineitems}

\index{nuclear\_repulsion\_energy() (in module globals.geom)}

\begin{fulllineitems}
\phantomsection\label{\detokenize{globals:globals.geom.nuclear_repulsion_energy}}\pysiglinewithargsret{\sphinxcode{globals.geom.}\sphinxbfcode{nuclear\_repulsion\_energy}}{\emph{geometry}}{}
Nuclear repulsion energy, hartrees

\end{fulllineitems}



\chapter{Indices and tables}
\label{\detokenize{index:indices-and-tables}}\begin{itemize}
\item {} 
\DUrole{xref,std,std-ref}{genindex}

\item {} 
\DUrole{xref,std,std-ref}{modindex}

\item {} 
\DUrole{xref,std,std-ref}{search}

\end{itemize}


\renewcommand{\indexname}{Python Module Index}
\begin{sphinxtheindex}
\def\bigletter#1{{\Large\sffamily#1}\nopagebreak\vspace{1mm}}
\bigletter{g}
\item {\sphinxstyleindexentry{globals.geom}}\sphinxstyleindexpageref{globals:\detokenize{module-globals.geom}}
\end{sphinxtheindex}

\renewcommand{\indexname}{Index}
\printindex
\end{document}